\documentclass[]{caesar_book}

% -- graphics --
\graphicspath{{./figures/}}

% -- language: German --
%\usepackage{csquotes}
\usepackage[english]{babel}

% -- biblatex --
\usepackage[backend = biber, style = philosophy-classic]{biblatex} % cleopatra matches the caesar book style
\addbibresource{library.bib} % the .bib file with the references
\AtEveryCitekey{\clearfield{title}} % display no article titles

% -- href compatibility --
%%
\definecolor{darkblue}{rgb}{0,0,.3}
\definecolor{darkgreen}{rgb}{0,.3,0}
\definecolor{graphblue}{rgb}{0,0,.7}
\definecolor{grey}{rgb}{.25,.25,.25}
\definecolor{orange}{rgb}{.8,.2,0}
\definecolor{green}{rgb}{0,0.5,0}
%\hypersetup{pdftex=true, linktocpage=true, colorlinks=true, breaklinks=true, linkcolor=darkblue, menucolor=darkblue, pagecolor=darkblue, urlcolor=darkblue, citecolor=darkgreen, frenchlinks=false}
%\renewcommand{\cite}[1]{\sidecite{#1}}

% -- misc --
\usepackage{lipsum}
\usepackage{nicefrac}

\usepackage{verbatim}

\newenvironment{snippet}%
{\endgraf\verbatim}%
{\endverbatim}

%\renewcommand{\sidestyle}{\raggedouter}

%%
%% -- jetzt gehts los --
%%

\title{Caesar}
\author{Andy Thomas}
\newcommand{\publisher}{Bielefeld University}


\begin{document}
% start
\frontmatter
% titlepage
\maketitlepage
% v.4 copyright page
\newpage
\begin{fullwidth}
~\vfill
\thispagestyle{empty} 
\setlength{\parindent}{0pt}
\setlength{\parskip}{\baselineskip}
\makeatletter
Copyright \copyright\ 2009 \@author
\makeatother
%\par{\publisher, Department of Physics}
%\par Schriftliche Habilitationsleistung
%\par\textit{layout edited \monthyear, content unchanged}
\end{fullwidth}
\newpage

% r.5 contents
\tableofcontents

% r.9 introduction
\cleardoublepage
%
\mainmatter
\chapter{Quick start guide}
This chapter is intended to quickly introduce the features of the caesar class. It basically comments the example file that comes with the class and explains how to use it. The class is written to be used with Lua-\LaTeX{} and Bib-\LaTeX{} and will not work otherwise. 

The numbers at the beginning correspond to the line numbers in the example file. First, the class is loaded with\\ 
\verb+1 \documentclass[]{caesar_book}+\\
The class is derived from the standard \LaTeX-book class and the square brackets allow the same parameters. The next task is to set the main language of the manuscript. \texttt{Babel} is used to set English (or any other language)\\
\verb+5 \usepackage[english]{babel}+\\
For German, \texttt{csquotes} is needed as well. Next, is the configuration of the references. \texttt{biblatex} is used in this case, it allows a lot of options to configure the exact look of the bibliography as well as the references in the text.\\
\verb+8 \usepackage[backend=biber,+\\
\verb+       style=philosophy-classic]{biblatex}+\\
\texttt{biber} is the natural companion of \texttt{biblatex}. Any style file can be used instead of \texttt{philosophy-classic}. Now, \texttt{biblatex} needs at east one resource file with the references.\\
\verb+10 \addbibresource{caesar_library.bib}+\\
In the example set, the name of the file is \texttt{library.bib}. Now, information about the book has to be added. The name of the author, the title and the publisher is automatically used to generate the title page.\\
\verb+14 \title{Caesar\\examples}+\\
\verb+15 \author{Andy Thomas}+\\
\verb+16 \newcommand{\publisher}{Bielefeld University}+\\
The document can start now.\\
\verb+18 \begin{document}+\\
The first pages of the book (the frontmatter) are not numbered, the numbering starts after the \texttt{mainmatter} macro, which is called after the generation of the title page.\\
\verb+20 \frontmatter+\\
\verb+22 \maketitlepage+\\
\verb+24 \mainmatter+\\
It is time for the first chapter. This in done in the usual way.\\
\verb+26 \chapter{Examples}+\\
A sidenote – a footnote in the margin – can be placed with the \texttt{sidenote} macro.\\
\verb+31 \sidenote{All ... necessary.}+
The citations are also placed in the margin of the document. This is done with the \texttt{sidecite} macro. The parameters mimic Biblatex's three parameters (2 optional) for citations.\\
\verb+33 \sidecite[Please see:][and more +\\
\verb+       work by Tufte.]{Tufte1990,Tufte2006}+\\
The sections are also started the common way.\\
\verb+34 \section{Figures}+\\
There are 3 different macros to place figures in the document.
\verb+37 \smallfigure{rectangle}{A ... margin.}+\\
\texttt{smallfigure} puts a figure in the margin. The first parameter is the filename which also serves as the label and the second parameter is the caption of the figure. A larger figure can be put in the text with \texttt{normalfigure} and a figure that spreads across the page width can be placed by using \texttt{largefigure}.

The same kind of macros are available for placing tables: \texttt{smalltable}, \texttt{normaltable} and \texttt{largetable}. The have 3 parameters each: reference, caption and the table code. \texttt{normaltable} and \texttt{largetable} also have the optional placing parameter, \texttt{[htbp]} is the default value.\\
\verb+53 \smalltable{table1}{A couple of numbers+\\
\verb+        in a table in the margin.}{%+\\
\verb+54  \begin{tabular}{lll}%+\\
\verb+55      A&B&C\\%+\\
\verb+56      0.50&0.47&0.48\\%+\\
\verb+57  \end{tabular}%+\\
\verb+58 }%+\\
Addtionally, there is a fullwidth environment that allows to fill text across the full page as well. However, it does not necessary work across page breaks.
\verb+79 \begin{fullwidth}+\\
\verb+80 Lorem ...+\\
\verb+81 \end{fullwidth}+\\
It might also overlap with the marginals, the sidenotes are not pushed up or down by \texttt{fullwidth}. 
The last macro allows a comment in the margin without a number. That can be achieved using
\verb+84 \margintext{It is also possible ... in the text.}+\\
Finally, the bibliography is placed using the \texttt{biblatex} syntax.
\verb+89 \printbibliography[heading=bibintoc]+\\
%
\chapter{Sidenotes package}
%
The \texttt{sidenotes} package contains the low level macros that do the actual typesetting
and figure placement in the margin. The package tries to allow typesetting of rich content in the margin.\sidenote{This is based on v0.81, dated 2011/11/29}
This includes text, but also figures, captions, tables and citations and is common in science textbooks such as Feyman's \textit{Lectures on Physics}.
%
\section{Usage}
%
The\margintext{sidenote} \texttt{sidenote} macro is very similar to
the footnote macro and tries to emulate its behavior. But like the name
says, the note is put in the margin, hence the name sidenote. It has the
same parameters as footnote:
\verb+\sidenote[number]{text}+. The sidenote moves up or down (floats)
to not overlap with other floats in the margin. All the sidenotes are subsequently numbered. The
first, optional parameter will manually change the numbering of the sidenote.

Sidenote tries to mimic the footnote behavior and tries to provide the same solutions. 
Sometimes it is not possible to directly call a sidenote macro, e.g. in particular environments. Then, 
you can also use\margintext{sidenotemark}  \verb+\sidenotemark[number]+ and\margintext{sidenotetext}
 \verb+\sidenotetext+ \verb+[number]{text}+ 
commands. \verb+\sidenotemark+ puts a mark at the current position. Then, outside of the environment 
that causes the trouble, it is possible the call \verb+\sidenotetext{text}+ to actually make the sidenote.
The first, optional parameter will manually change the numbering of the sidenotes.

You can use\margintext{sidestyle} \verb+\renewcommand{\sidestyle}{something}+ 
if you want to change the font, text size, text color or something else of the sidenotes.
It it initialized with \verb+\footnotesize+. It is used as a prefix of the sidenotetext and sidetext.

The macro\margintext{sidecite} \verb+\sidecite+ puts a citation in the margin. It uses the biblatex package or bibtex,
load sidecite with the option [bibtex] for the latter.
The macro has the same set of parameters.
\verb+\sidecite[prenote][postnote]{key}+ for biblatex and \verb+\sidecite{key}+ for bibtex.
The behavior is the same as in \verb+\sidenote+ and auto floating. 
For post- and prenote please refer to the biblatex manual, the parameters are directly passed
to biblatex.

The\margintext{sidecaption} \verb+\sidecaption+ macro can be used if the caption of a figure or table 
is supposed to be in the margin. Please note, that the formatting is done by the 
caption package. 

The\margintext{sidefigure} sidefigure environment puts a figure and its caption in the margin. Instead of  
\verb+\begin{figure}+ use \verb+\begin{sidefigure}+. Please note, that 
the use of caption before \verb+\includegraphics+ puts the caption in line with the
top of the figure
and the use after \verb+\includegraphics+ puts it in line with the bottom of the actual figure.

The sidetable\margintext{sidetable} environment works similarly, but with table environments. Use \verb+\begin{sidetable}+ instead
of \verb+\begin{table}+.

\section{Technical notes and further macros}
%
Sometimes it is useful to put text in the margin without a mark in the text. However, this in not formatted by \texttt{sidestyle} and can be achieved
with\margintext{marginpar} \verb+\marginpar{text}+. The \verb+\sidecaption+ macro relies on the
marginnote\margintext{marginnote} package by Markus Kohm. 
 
When writing the package, we tried to be as general as possible. Someone can e.g.\ use sidenotes mixed with
footnotes. Also, the package tries to provide only functionality and does not know anything about formatting
such as margin text size, color or anything else. Only \verb+\sidestyle+ was added for convenience. If you are
looking for a package that provides formatting defaults as well you might want to look into caesar style that accompanies this package.

\section{Required packages}
\begin{description}
     \item[marginnote]
        supports another command to create notes in the margin. The notes are not floats , but can be manually shifted up or down. 
     \item[caption]
        is used to set figure and table captions in the margin and to allow formatting of these captions.
	\item[xifthen] is used to test for empty, optional arguments.  
\end{description}%

\end{document}

\chapter{Shaping the page}
%
The page layout utilizes one of the suggestions made by Robert Bringhurst in his typography book.\cite{xxx} One main text column and an ample margin column are used. The margin can hold sidenotes, small figures, the references and any other additional information. 
%
\normalfigure[h]{page_layout2}{}
%
This is achieved in the \LaTeX -class with the following code:
\begin{snippet}
\RequirePackage[paperwidth=170mm, 
paperheight=240mm, left=40pt, top=40pt, 
textwidth=280pt, marginparsep=20pt, 
marginparwidth=100pt, textheight=560pt, 
footskip=40pt]
{geometry}%
\end{snippet}
The corresponding ConTeXt-code is:
\begin{snippet}
% setup the page format
\definepapersize[wissenschaft][width=170mm,height=240mm]
% use the new page format
\setuppapersize[wissenschaft]
% shape the page layout
\setuplayout[topspace=40pt,
			header=0pt,
			headerdistance=0pt,
			backspace=40pt,
			leftmargin=0pt,
			width=280pt,
			height=560pt,
			rightmargindistance=20pt,
			rightmargin=100pt,
			footer=0pt]
\setuppagenumbering[alternative=doublesided]
\end{snippet}

\end{document}
