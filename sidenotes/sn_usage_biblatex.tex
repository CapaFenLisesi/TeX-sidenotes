\documentclass{book}

\usepackage{sidenotes}

% provide some simple layout
\usepackage[paperwidth=210mm, paperheight=297mm, left=50pt, top=50pt, textwidth=345pt, marginparsep=25pt, marginparwidth=124pt, textheight=692pt, footskip=50pt]{geometry}
\pagestyle{empty}
\captionsetup{width=\marginparwidth, font=footnotesize, skip = 0pt}
\renewcommand{\thefootnote}{\fnsymbol{footnote}}

\usepackage[backend = biber, citestyle = numeric]{biblatex}
\addbibresource{sn.bib}
\usepackage{marginfix}
\begin{document}
%
\chapter{Usage of the \textbf{sidenotes} package}
The package provides a number of macros to typeset rich content in the margin. Please read the output of this \LaTeX{} file and look at the input file while doing so. We start with the \verb+\sidenote+ macro. 

\section{\texttt{\textbackslash sidenote[number]\{text\}}}
It is very similar to \verb+\footnote+ and tries to emulate its behavior.\sidenote{But like the name says, the note is put in the margin, hence the name sidenote.} It has the same parameters as footnote. The sidenote moves up or down (floats) to not overlap with other floats in the margin.\sidenote{Note how this paragraph is slightly moved down.} All the sidenotes are subsequently numbered. The first, optional parameter will manually change the numbering of the sidenote.\sidenote[666]{This note is evil.}

\section{\texttt{\textbackslash sidenotemark[number]}}
Sidenote tries to mimic the footnote behavior and provides the same solutions. Sometimes it is not possible to directly call a sidenote macro, e.g. in particular environments. Then, you can also use \verb+\sidenotemark+ macro.\sidenotemark{} If a number is given, that number is used.\sidenotemark[99]{} This puts a mark at the current position. 

\section{\texttt{\textbackslash sidenotetext[number]\{text\}}}
Afterwards,\sidenotetext{Please note, how the numbering still counts from the last usual call} outside of the environment that causes the trouble, it is possible the call \verb+\sidenotetext{text}+ to actually make the sidenote. The optional parameter will again manually change the numbering of the sidenote.\sidenotetext[93]{Then, you have to make sure the number match!}

\section{\texttt{\textbackslash sidestyle}}
\renewcommand{\sidestyle}{\raggedright\footnotesize}
You can use \verb+\renewcommand{\sidestyle}{something}+, if you want to change the font, text size, text color or something else of the sidenotes.\sidenote{Here, we changed sidestyle to raggedright.}
\renewcommand{\sidestyle}{\footnotesize\em}
The default value is \verb+\footnotesize+, \verb+\sidestyle+ is the prefix for all sidenotes.\sidenote{Here, we changed to \texttt{\textbackslash em}} Please note that the values do not add up, \verb+\sidestyle+ is overwritten by every use of \verb+\renewcommand{\sidestyle}{...}+.%
\renewcommand{\sidestyle}{\footnotesize}%
\sidenote{Back to default behavior.}

\section{\texttt{\textbackslash sidecite[prefix][postfix]\{citekey\}}}
The macro \verb+\sidecite+ puts a citation in the margin. The macro has the same set of parameters as biblatex. We add a citation here as an example.\sidecite[see:][Nobel prize 1972]{nobel:bcs} The behavior is the same as in \verb+\sidenote+ and auto floating. For post- and prenote please refer to the biblatex manual, the parameters are directly passed to biblatex.

\section{\texttt{\textbackslash sidecaption\{captiontext\}}}
The \verb+\sidecaption+ macro can be used if the caption of a figure or table is supposed to be in the margin. Please note, that the formatting is done by the caption package. 
\begin{figure}[h]
\sidecaption{I am a caption, but in the margin.}
\fbox{\rule{\textwidth}{0mm}\rule{0mm}{1.5cm}} % \includegraphics{image}
\end{figure}
Simply use \verb+\sidecaption+ instead of \verb+\caption+ in the figure environment. 

\section{\texttt{sidefigure}}
The sidefigure environment puts a figure and its caption in the margin. Instead of \verb+\begin{figure}+ use \verb+\begin{sidefigure}+. Please note, that the use of caption before \verb+\includegraphics+ puts the caption in line with the top of the figure and the use after \verb+\includegraphics+ puts it in line with the bottom of the actual figure. The formatting is done by the caption package.
\begin{sidefigure}
\fbox{\rule{0mm}{1.2cm}\rule{\marginparwidth}{0mm}} % \includegraphics{image}
\caption{This is a figure with its caption in the margin.}
\end{sidefigure}

\section{\texttt{sidetable}}
The sidetable environment works similarly to sidefigure, but with table environments. Use \verb+\begin{sidetable}+ instead of \verb+\begin{table}+. The formatting is also done with help of the caption package.
\begin{sidetable}
  \centering
 % \fontfamily{ppl}\selectfont
  \begin{tabular}{lllll}
  \hline
     Hg&Sn&Pb&Cd&Tl \\
     \hline
    0.50&0.47&0.48&0.5&0.5\\
  \end{tabular}
  \caption{This is a table in the margin}
\end{sidetable}

\section{Technical notes and further macros} 
Sometimes it is useful to put text in the margin without a mark in the text. This can be achieved with \verb+\marginpar{text}+. 

The \verb+\sidecaption+ macro relies on the marginnote package by Markus Kohm and is used by the sidenotes package automatically.

When writing the package, we tried to be as general as possible. Someone can e.g.\ use sidenotes mixed with footnotes.\footnote{Here, we have a footnote as well. The marker was changed to symbols to avoid confusion with the same numbers as sidenotes.} Also, the package tries to provide only functionality and does not know anything about formatting such as margin text size, color or anything else. Only \verb+\sidestyle+ was added for convenience. 

\end{document}
